\documentclass[12pt, letterpaper, twoside]{article}
\usepackage{amsmath}
\usepackage{braket}

\title{QCQI Chapter 4 Exercises}
\author{Felix Tripier}

\newcommand{\iu}{{i\mkern1mu}}
\newcommand{\Hadamard}{\frac{1}{\sqrt{2}}
\begin{bmatrix}
    1 & 1 \\
    1 & -1
\end{bmatrix}}
\newcommand{\Rz}{\begin{bmatrix}
        e^{-\iu\theta/2} & 0 \\
        0 & e^{\iu\theta/2}
    \end{bmatrix}}
\newcommand{\halftheta}{\frac{\theta}{2}}
\newcommand{\Rx}{\begin{bmatrix}
    \cos{\halftheta} & -\iu\sin{\halftheta} \\
    -\iu\sin{\halftheta} & \cos{\halftheta}
\end{bmatrix}}
\newcommand{\twoXtwo}[4]{\begin{bmatrix}
    #1 & #3 \\
    #2 & #4
\end{bmatrix}}
\newcommand{\Rn}{\twoXtwo{
    \cos{\halftheta} - \iu \sin{\halftheta}n_z
    }{
        \iu \sin{\halftheta}(n_x + \iu n_y)
    }{
        \iu \sin{\halftheta}(n_x - \iu n_y)
    }{
        \cos{\halftheta} + \iu \sin{\halftheta}n_z
        }}


\begin{document}

\maketitle
4.2: Let $x$ be a real number and $A$ a matrix such that $A^2 = I$. Show that

$$
	\exp(\iu Ax) = \cos(x)I + \iu \sin(x)A
$$

Answer:

\begin{subequations}
	\begin{align}
		\exp(\iu Ax) = \sum_{n=1}^{\infty} \frac{1}{n!} (\iu Ax)^n                                \\
		= \sum_{n=1}^{\infty} \frac{1}{(2n)!} (\iu Ax)^{2n} + \frac{1}{(2n + 1)!} (\iu Ax)^{2n+1} \\
		= I\sum_{n=1}^{\infty} \frac{(-1)^n}{(2n)!} x^{n} + \frac{1}{(2n + 1)!} (\iu Ax)^{2n+1}   \\
		= I\sum_{n=1}^{\infty} \frac{(-1)^n}{(2n)!} x^{n} + \iu A\frac{(-1)^n}{(2n + 1)!} x^n     \\
		= \cos(x)I + \iu \sin(x)A
	\end{align}
\end{subequations}

4.3: Show that, up to a global phase, the $\pi/8$ gate satisfies $T = R_z(\pi/4)$

Answer:
\[
	\pi/8 \text{ gate} =
	exp(\iu\pi/8)
	\begin{bmatrix}
		e^{-\iu \pi/8} & 0              \\
		0              & e^{-\iu \pi/8}
	\end{bmatrix}
\]
and
\[
	R_z(\pi/4) =
	\begin{bmatrix}
		e^{-\iu \pi/8} & 0              \\
		0              & e^{-\iu \pi/8}
	\end{bmatrix}
\]

$exp(\iu\pi/8)$ is just a global phase factor dude.


4.4: Express the Hadamard gate $H$ as a product of $R_x$ and $R_z$ rotations and $e^{i\varphi}$

Answer:

\begin{subequations}
	\begin{align}
		\text{Hadamard} = \Hadamard           \\
		R_z(\theta) = \Rz                     \\
		R_x(\theta) = \Rx                     \\
		R_z(-2\pi) = \twoXtwo{1}{0}{0}{-1}    \\
		R_x(\pi) = \twoXtwo{0}{-\iu}{-\iu}{0} \\
		e^{\iu\cdot\frac{5\pi}{2}} \cdot R_x(\pi/2) \cdot R_z(\pi/2) \cdot R_x(\pi/2) = \Hadamard
	\end{align}
\end{subequations}
The insight here is that this series of rotations is equivalent (within a global phase factor) to the Hadamard's more obvious set of rotations: 90 degrees around the y axis followed by 180 degrees around the x axis.

4.5 Prove that $(\hat{n} \cdot \sigma)^2 = I$ and use this to verify this equation:
$$
	R_{\hat{n}}(\theta) \equiv \exp(-\iu \theta \hat{n} \cdot \sigma/2) = \cos(\frac{\theta}{2})I - \iu \sin(\frac{\theta}{2})(n_xX + n_yY + n_zZ)
$$

Okay, proof by equational reasoning:

$\hat{n}$ is a unit vector so $n_z^2 + n_x^2 + n_y^2 = 1$.

\begin{subequations}
	\begin{align}
		(\hat{n} \cdot \sigma)^2 = (\twoXtwo{0}{n_x}{n_x}{0} + \twoXtwo{0}{-\iu n_y}{-\iu n_y}{0} + \twoXtwo{n_z}{0}{0}{-n_z})^2 \\
		= (\twoXtwo{n_z}{n_x + \iu n_y}{n_x - \iu n_y}{n_z})^2
		= \twoXtwo{n_z^2 + n_x^2 + n_y^2}{2n_z(n_x + \iu n_y)}{2n_z(n_x - \iu n_y)}{n_z^2 + n_x^2 + n_y^2}                       \\
		= \twoXtwo{1}{2n_z(n_x + \iu n_y)}{2n_z(n_x - \iu n_y)}{1}
	\end{align}
\end{subequations}
Hermitian operators must be equal to their adjoint and so
$$
	2n_z(n_x + \iu n_y) = 2n_z(n_x - \iu n_y)
$$

We'll call this term "scooby-doo".

The determinant of a rotation matrix must be 1, and so we have
\begin{subequations}
	\begin{align}
	1 - \text{scooby-doo}^2 = 1 \\
	- \text{scooby-doo}^2 = 0 \\
	\text{scooby-doo} = 0
\end{align}
\end{subequations}

and therefore

$$
	\twoXtwo{1}{\text{scooby-doo}}{\text{scooby-doo}}{1} = \twoXtwo{1}{0}{0}{1} = I
$$

Having done this convuluted and scooby-dooful proof, I now realize that since all quantum operators must be Hermitian and unitary, then $R_{\hat{n}}^2$ must trivially be equal to $I$ by:
\begin{subequations}
	\begin{align}
R_{\hat{n}}^\dagger R_{\hat{n}} = I \text{ by being unitary} \\
R_{\hat{n}}^\dagger = R_{\hat{n}} \text{ by being Hermitian} \\
R_{\hat{n}}^2 = I
\end{align}
\end{subequations}

4.6: I'm getting kinda tired so I'm not gonna write the full question for this one. It's so long dude. They want me to prove that $R_{\hat{n}}(\theta)$ rotates a Bloch vector around the $\hat{n}$ axis.

For a state vector $a\ket{0} + b\ket{1}$
\begin{subequations}
\begin{align}
R_n(\theta) = \Rn \\
R_n{\theta}(a\ket{0} + b\ket{1}) = e^{-\iu \halftheta}n_z a\ket{0} + \iu \sin{\halftheta}(n_x - \iu n_y)b\ket{1} + \iu \sin{\halftheta}(n_x + \iu n_y)a\ket{0} +  e^{\iu \halftheta}n_z b\ket{1} \\
\frac{\partial R_n{\theta}(a\ket{0} + b\ket{1})}{\partial a\ket{0}} = e^{-\iu \halftheta}n_z + \iu \sin{\halftheta}(n_x + \iu n_y) \\
\frac{\partial R_n{\theta}(a\ket{0} + b\ket{1})}{\partial b\ket{0}} = e^{\iu \halftheta}n_z + \iu \sin{\halftheta}(n_x - \iu n_y) \\
\frac{\partial R_n{\theta}(a\ket{0} + b\ket{1})}{\partial a\ket{0}} - \frac{\partial R_n{\theta}(a\ket{0} + b\ket{1})}{\partial b\ket{0}} = 2\iu \sin{\halftheta} n_z - 2\sin{\halftheta} n_y
\end{align}
\end{subequations}

which shows up in the polar coordinates basis $(\theta, \varphi)$ as
\begin{subequations}
\begin{align}
a = \cos{\halftheta} \text{ by definition of the Bloch vector} \\
b = e^{\iu \varphi} \sin{\halftheta} \text{ by definition of the Bloch vector} \\
\theta = 2\arccos{a} \\
\varphi = -\iu \ln{\frac{b}{\sin{\halftheta}}}
\end{align}
\end{subequations}


\end{document}