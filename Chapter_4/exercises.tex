\documentclass[12pt, letterpaper, twoside]{article}
\usepackage{amsmath}

\title{QCQI Chapter 4 Exercises}
\author{Felix Tripier}

\newcommand{\iu}{{i\mkern1mu}}
\newcommand{\Hadamard}{\frac{1}{\sqrt{2}}
\begin{bmatrix}
    1 & 1 \\
    1 & -1
\end{bmatrix}}
\newcommand{\Rz}{\begin{bmatrix}
        e^{-\iu\theta/2} & 0 \\
        0 & e^{\iu\theta/2}
    \end{bmatrix}}
\newcommand{\halftheta}{\frac{\theta}{2}}
\newcommand{\Rx}{\begin{bmatrix}
    \cos{\halftheta} & -\iu\sin{\halftheta} \\
    -\iu\sin{\halftheta} & \cos{\halftheta}
\end{bmatrix}}
\newcommand{\twoXtwo}[4]{\begin{bmatrix}
    #1 & #3 \\
    #2 & #4
\end{bmatrix}}


\begin{document}

\maketitle
4.2: Let $x$ be a real number and $A$ a matrix such that $A^2 = I$. Show that

$$
	\exp(\iu Ax) = \cos(x)I + \iu \sin(x)A
$$

Answer:

\begin{subequations}
\begin{align}
	\exp(\iu Ax) = \sum_{n=1}^{\infty} \frac{1}{n!} (\iu Ax)^n                                \\
	= \sum_{n=1}^{\infty} \frac{1}{(2n)!} (\iu Ax)^{2n} + \frac{1}{(2n + 1)!} (\iu Ax)^{2n+1} \\
	= I\sum_{n=1}^{\infty} \frac{(-1)^n}{(2n)!} x^{n} + \frac{1}{(2n + 1)!} (\iu Ax)^{2n+1}   \\
	= I\sum_{n=1}^{\infty} \frac{(-1)^n}{(2n)!} x^{n} + \iu A\frac{(-1)^n}{(2n + 1)!} x^n     \\
	= \cos(x)I + \iu \sin(x)A
\end{align}
\end{subequations}

4.3: Show that, up to a global phase, the $\pi/8$ gate satisfies $T = R_z(\pi/4)$

Answer:
\[
    \pi/8 \text{ gate} =
    exp(\iu\pi/8)
\begin{bmatrix}
    e^{-\iu \pi/8} & 0 \\
    0 & e^{-\iu \pi/8}
\end{bmatrix}
\]
and
\[
    R_z(\pi/4) =
\begin{bmatrix}
    e^{-\iu \pi/8} & 0 \\
    0 & e^{-\iu \pi/8}
\end{bmatrix}
\]

$exp(\iu\pi/8)$ is just a global phase factor dude.


4.4: Express the Hadamard gate $H$ as a product of $R_x$ and $R_z$ rotations and $e^{i\varphi}$

Answer:

\begin{subequations}
\begin{align}
    \text{Hadamard} = \Hadamard \\
    R_z(\theta) = \Rz\\
    R_x(\theta) = \Rx\\
    R_z(-2\pi) = \twoXtwo{1}{0}{0}{-1}\\
    R_x(\pi) = \twoXtwo{0}{-\iu}{-\iu}{0}\\
    e^{\iu*\frac{5\pi}{2}} * R_x(\pi/2) * R_z(\pi/2) * R_x(\pi/2) = \Hadamard
\end{align}
\end{subequations}
The insight here is that this series of rotations is equivalent (within a global phase factor) to the Hadamard's more obvious set of rotations: 90 degrees around the y axis followed by 180 degrees around the x axis.

\end{document}