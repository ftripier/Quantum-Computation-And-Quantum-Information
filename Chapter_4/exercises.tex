\documentclass[12pt, letterpaper, twoside]{article}
\usepackage{amsmath}
\usepackage{braket}
\usepackage{amssymb}

\title{QCQI Chapter 4 Exercises}
\author{Felix Tripier}

\newcommand{\iu}{{i\mkern1mu}}
\newcommand{\Hadamard}{\frac{1}{\sqrt{2}}
\begin{bmatrix}
    1 & 1 \\
    1 & -1
\end{bmatrix}}
\newcommand{\Rz}{\begin{bmatrix}
        e^{-\iu\theta/2} & 0 \\
        0 & e^{\iu\theta/2}
    \end{bmatrix}}
\newcommand{\halftheta}{\frac{\theta}{2}}
\newcommand{\Rx}{\begin{bmatrix}
    \cos{\halftheta} & -\iu\sin{\halftheta} \\
    -\iu\sin{\halftheta} & \cos{\halftheta}
\end{bmatrix}}
\newcommand{\twoXtwo}[4]{\begin{bmatrix}
    #1 & #3 \\
    #2 & #4
\end{bmatrix}}
\newcommand{\Rn}{\twoXtwo{
    \cos{\halftheta} - \iu \sin{\halftheta}n_z
    }{
        \iu \sin{\halftheta}(n_x + \iu n_y)
    }{
        \iu \sin{\halftheta}(n_x - \iu n_y)
    }{
        \cos{\halftheta} + \iu \sin{\halftheta}n_z
        }}
\newcommand{\pauliX}{\twoXtwo{0}{1}{1}{0}}
\newcommand{\pauliY}{\twoXtwo{0}{\iu}{-\iu}{0}}
\newcommand{\pauliZ}{\twoXtwo{1}{0}{0}{-1}}



\begin{document}

\maketitle
4.2: Let $x$ be a real number and $A$ a matrix such that $A^2 = I$. Show that

$$
	\exp(\iu Ax) = \cos(x)I + \iu \sin(x)A
$$

Answer:

\begin{subequations}
	\begin{align}
		\exp(\iu Ax) = \sum_{n=1}^{\infty} \frac{1}{n!} (\iu Ax)^n                                \\
		= \sum_{n=1}^{\infty} \frac{1}{(2n)!} (\iu Ax)^{2n} + \frac{1}{(2n + 1)!} (\iu Ax)^{2n+1} \\
		= I\sum_{n=1}^{\infty} \frac{(-1)^n}{(2n)!} x^{n} + \frac{1}{(2n + 1)!} (\iu Ax)^{2n+1}   \\
		= I\sum_{n=1}^{\infty} \frac{(-1)^n}{(2n)!} x^{n} + \iu A\frac{(-1)^n}{(2n + 1)!} x^n     \\
		= \cos(x)I + \iu \sin(x)A
	\end{align}
\end{subequations}

4.3: Show that, up to a global phase, the $\pi/8$ gate satisfies $T = R_z(\pi/4)$

Answer:
\[
	\pi/8 \text{ gate} =
	exp(\iu\pi/8)
	\begin{bmatrix}
		e^{-\iu \pi/8} & 0              \\
		0              & e^{-\iu \pi/8}
	\end{bmatrix}
\]
and
\[
	R_z(\pi/4) =
	\begin{bmatrix}
		e^{-\iu \pi/8} & 0              \\
		0              & e^{-\iu \pi/8}
	\end{bmatrix}
\]

$exp(\iu\pi/8)$ is just a global phase factor dude.


4.4: Express the Hadamard gate $H$ as a product of $R_x$ and $R_z$ rotations and $e^{i\varphi}$

Answer:

\begin{subequations}
	\begin{align}
		\text{Hadamard} = \Hadamard           \\
		R_z(\theta) = \Rz                     \\
		R_x(\theta) = \Rx                     \\
		R_z(-2\pi) = \twoXtwo{1}{0}{0}{-1}    \\
		R_x(\pi) = \twoXtwo{0}{-\iu}{-\iu}{0} \\
		e^{\iu\cdot\frac{5\pi}{2}} \cdot R_x(\pi/2) \cdot R_z(\pi/2) \cdot R_x(\pi/2) = \Hadamard
	\end{align}
\end{subequations}
The insight here is that this series of rotations is equivalent (within a global phase factor) to the Hadamard's more obvious set of rotations: 90 degrees around the y axis followed by 180 degrees around the x axis.

4.5 Prove that $(\hat{n} \cdot \sigma)^2 = I$ and use this to verify this equation:
$$
	R_{\hat{n}}(\theta) \equiv \exp(-\iu \theta \hat{n} \cdot \sigma/2) = \cos(\frac{\theta}{2})I - \iu \sin(\frac{\theta}{2})(n_xX + n_yY + n_zZ)
$$

All quantum operators must be Hermitian and unitary, so then $R_{\hat{n}}^2$ must trivially be equal to $I$ by:
\begin{subequations}
	\begin{align}
R_{\hat{n}}^\dagger R_{\hat{n}} = I \text{ by being unitary} \\
R_{\hat{n}}^\dagger = R_{\hat{n}} \text{ by being Hermitian} \\
R_{\hat{n}}^2 = I
\end{align}
\end{subequations}

4.6: I'm getting kinda tired so I'm not gonna write the full question for this one. It's so long dude. They want me to prove that $R_{\hat{n}}(\theta)$ rotates a Bloch vector around the $\hat{n}$ axis.

$$
R_n(\theta) = \Rn
$$

One observes that this matrix has the effect of changing the $\ket{0}\ket{1}$ basis vectors into
$$
\ket{0'} = (\cos{\halftheta} - \iu \sin{\halftheta}n_z)\ket{0} + \iu\sin{\halftheta}(n_x + \iu n_y) \ket{1}
$$
$$
\ket{1'} = (\cos{\halftheta} + \iu \sin{\halftheta}n_z)\ket{1} +  \iu\sin{\halftheta}(n_x - \iu n_y) \ket{0}
$$
My proof follows by proving these two vectors form an orthonormal basis that traces out a unit sphere in $\mathbb{R}^3$. In the interest of saving time, I didn't make it fully rigorous.

Both new vectors must be of norm one, because the rotation matrix is defined as
$$
R_{\hat{n}}(\theta) = exp(-\iu \theta \hat{n} * \sigma/2)
$$

Because the rotation operator is defined from a matrix exponential, and is Hermitian and unitary, it must have a basis of orthonormal vectors. Since the components of base vector are defined by a linear combination of four variables with only two equations and trigonometric functions, with an imaginary summand, their domain is in the unit circle of $\mathbb{C}$.

A vector of two complex numbers is equal in cardinality to $\mathbb{R}^4$, and since each component is in the unit circle of $\mathbb{C}$, the unit sphere is a subset of the values. There exists a bijection to the bloch sphere, by the axioms of set theory.

4.7 Show that $XYX = -Y$ and use this to prove that $X R_y(\theta)X = R_y(-\theta)$
\begin{subequations}
\begin{align}
XYX = \pauliX\pauliY\pauliX\\
\pauliX\twoXtwo{-\iu}{0}{0}{\iu}\\
= \twoXtwo{0}{-\iu}{\iu}{0} = -Y\\
XR_y(\theta)X = X(\cos{\halftheta}I - \iu \sin{\halftheta}Y)X\\
= \cos{\halftheta}XX - \iu \sin{\halftheta}XYX\\
= \cos{\halftheta}I +\iu \sin{\halftheta}Y\\
= R_y(-\theta)
\end{align}
\end{subequations}

\end{document}